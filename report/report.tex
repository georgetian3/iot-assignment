本次作业使用 Python 3 实现所有功能,均存放在 \code{4.py} 中。
\section{依赖}

\begin{itemize}
    \item Python 3
    \item 若干个 Python 包(见 \code{README.md} 以及 \code{requirements.txt})
\end{itemize}

\section{实现}

\subsection{\code{dft}}

函数 \code{dft} 实现了离散傅立叶变换算法,其公式为:$$X_k=\sum_{n=0}^{N-1}x_ne^{\frac{2\pi ikn}{N}}$$

通过使用 \code{test_dft} 函数将 \code{dft} 的输出与 \code{numpy.fft.fft} 比较,得出两个函数的输出是非常相似的。由于 NumPy 的实现使用 FFT,所以会存在一定的偏差,但偏差的模小于 $1^{-10}$。

\subsection{\code{dft_to_freq_amp}}

此函数使用 \code{dft} 返回系数以及采样率计算每个频率对应的幅度,其公式为:

\begin{align*}
    f_i&=\frac{if_s}{N}\\
    A_i&=\frac{N|c_i|}{f_s}
\end{align*}

其中 $f_s$ 为采样率,$N$ 为采样点的个数,而 $c_i$ 是 DFT 返回的第 $i$ 个系数。

\subsection{其他}

除了以上两个主要函数之外,\code{read} 返回 \code{res.wav} 的采样点以及采样率。\code{a}、\code{b}、\code{c} 三个函数分别返回第一题中的三个函数的 $N$ 个值。\code{graph_freq_amp} 与 \code{spectrogram} 分别绘制频谱图以及时频图。\code{spectrogram} 依赖于 \code{matplotlib.pyplot.specgram}。\code{extend} 使用 \code{0} 将数据填充到原始长度的十倍。

\newpage

\section{实验}

所有实验可以通过调用 \code{run} 函数而运行。其中绘制补零后的 \code{res.wav} 会需要比较多的时间。

\begin{enumerate}
    \item 三个函数的频谱图如下所示:
    \begin{figure}[h!]
        \centering
        \begin{subfigure}{0.3\linewidth}
            \centering
            \scalebox{0.4}{\input{function a, N = 32.pgf}}
        \end{subfigure}
        \begin{subfigure}{0.3\linewidth}
            \centering
            \scalebox{0.4}{\input{function a, N = 128.pgf}}
        \end{subfigure}
        \begin{subfigure}{0.3\linewidth}
            \centering
            \scalebox{0.4}{\input{function a, N = 1024.pgf}}
        \end{subfigure}
        \\
        \begin{subfigure}{0.3\linewidth}
            \centering
            \scalebox{0.4}{\input{function b, N = 32.pgf}}
        \end{subfigure}
        \begin{subfigure}{0.3\linewidth}
            \centering
            \scalebox{0.4}{\input{function b, N = 128.pgf}}
        \end{subfigure}
        \begin{subfigure}{0.3\linewidth}
            \centering
            \scalebox{0.4}{\input{function b, N = 1024.pgf}}
        \end{subfigure}
        \\
        \centering
        \begin{subfigure}{0.3\linewidth}
            \centering
            \scalebox{0.4}{\input{function c, N = 32.pgf}}
        \end{subfigure}
        \begin{subfigure}{0.3\linewidth}
            \centering
            \scalebox{0.4}{\input{function c, N = 128.pgf}}
        \end{subfigure}
        \begin{subfigure}{0.3\linewidth}
            \centering
            \scalebox{0.4}{\input{function c, N = 1024.pgf}}
        \end{subfigure}
    \end{figure}

\newpage

    \item 声音信号文件分析
    
    \begin{enumerate}
        \item 分析信号频率组成
        \begin{figure}[h!]
            \centering
            \scalebox{0.7}{\input{res.wav, no padding.pgf}}
        \end{figure}
        \item 补零
        \begin{figure}[h!]
            \centering
            \scalebox{0.7}{\input{res.wav, 10x padding.pgf}}
        \end{figure}

        原始的声波可能是周期性比较强的信号,因此可以使用少数正弦波的叠加而产生。但补零操作会破坏此周期性,因此需要更多正弦波的叠加而产生。则补零后的频谱图的线的密度较高,因为如果放大可以看到许多小的波。

        \newpage

        \item 时频分析

        \begin{figure}[h!]
            \centering
            \begin{subfigure}{0.3\linewidth}
                \centering
                \scalebox{0.4}{\input{res.wav, N = 200.pgf}}
            \end{subfigure}
            \begin{subfigure}{0.3\linewidth}
                \centering
                \scalebox{0.4}{\input{res.wav, N = 300.pgf}}
            \end{subfigure}
            \begin{subfigure}{0.3\linewidth}
                \centering
                \scalebox{0.4}{\input{res.wav, N = 400.pgf}}
            \end{subfigure}
            \\
            \begin{subfigure}{0.3\linewidth}
                \centering
                \scalebox{0.4}{\input{res.wav, N = 500.pgf}}
            \end{subfigure}
            \begin{subfigure}{0.3\linewidth}
                \centering
                \scalebox{0.4}{\input{res.wav, N = 600.pgf}}
            \end{subfigure}
            \begin{subfigure}{0.3\linewidth}
                \centering
                \scalebox{0.4}{\input{res.wav, N = 700.pgf}}
            \end{subfigure}
            \\
            \begin{subfigure}{0.3\linewidth}
                \centering
                \scalebox{0.4}{\input{res.wav, N = 800.pgf}}
            \end{subfigure}
            \begin{subfigure}{0.3\linewidth}
                \centering
                \scalebox{0.4}{\input{res.wav, N = 900.pgf}}
            \end{subfigure}
            \begin{subfigure}{0.3\linewidth}
                \centering
                \scalebox{0.4}{\input{res.wav, N = 1000.pgf}}
            \end{subfigure}
        \end{figure}
    \end{enumerate}

    当窗口长度增加,两条绿线(表示频率为 8Hz 以及 20Hz 的波)越来越细,表示 DFT 更加准确地将原始波分解成两个正弦波。但是其他三条蓝线(可能是两个正弦波的谐波)也变的更加明显。

\end{enumerate}